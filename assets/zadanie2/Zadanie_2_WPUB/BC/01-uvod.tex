\section{Úvod}
% Text je súčasťou života každého človeka, je to jednou z dôležitých prostriedkov komunikácie myšlienok a informácií. 

% Mnoho ľudských činností v sebe zahŕňa buď vstrebávanie text zvyčajne čítaním alebo tvorbu text. Tieto činnosti môžu byť veľmi pracné a nie vždy dostatočné na obetovanie množstva času a peňazí, ktoré by bolo potrebné na ich realizáciu.

% V dnešnej dobe informačnej preťaženosti, prístupu k internetu a veľkému množstvu textu je dôležité mať spôsob ako sa s veľkým množstvom text vysporiadať a oddeliť užitočné od zbytočného.

% Aj ľudia aj stroje by profitovali z automatizácie týchto činností. Zautomatizovanie takýchto činnosti sa zaoberá NLG alebo generovanie prirodzeného jazyka.

% Môžeme nájsť viacero definícií NLG jednou z klasických je, že NLG je jednou z oborov umelej inteligencie a NLP, ktorej úlohou je zvyčajne z nejazyčných vstupov vygenerovať text pre rôzne domény a oblasti ľudských činnosti \cite{reiter_dale_2000_buildingnlgsystems}.

% V poslednej dobre sa však na rôzne úlohy hlavne text-to-text používajú prístupy, ktorých vstupom nie sú len dáta ale aj text\cite{Nallapati_2016_sum_s2s, Nisioi_2017_simpl_expl, Mallinson_2017_para_w_mt}. Rozdiel medzi touto prvotnou definíciou NLG a reálnymi aplikáciami v súčasnosti značí, že NLG prešlo počas posledných dvoch desaťročí mnohými zmenami.
% % \cite{ref1, ref2...} 

% Vo všeobecnosti ide o generovanie textu, ktorý je priamo určený na čítanie pre ľudí. Táto definícia je všeobecná a preto má NLG široké využitie. Niektoré príklady využitia NLG zahŕňajú napr. generovanie článkov do novín ako napríklad futbalové reportáže\cite{vanderlee_krahmer_wubben_2017_PASS}, zdravotnícke správy, ale aj všeobecnejšie, nezávislé od domény využitia ako simplifikácia alebo sumarizácia textu, generovanie parafráz, prekladanie textov\cite{gatt_2018_survey}.



Zrak,hmat a sluch sú tromi našimi najdôležitejšími vnemami, pomocou ktorých prijímame veľké množstvo informácií a bez ktorých by sme sa ťažko v živote zaobišli. Práve preto text, či už vo viditeľnej, hmatateľnej alebo zvukovej podobe predstavuje dôležité médium na prenášanie, komunikáciu a organizovanie dát a informácií, ktoré má hlavne v dnešnej dobe internetu a sociálnych sietí veľmi dôležitú rolu.

Čím viac sa spoločnosť blíži k viac digitálnej kde informácií je veľmi veľa bude treba tvoriť ale aj vstrebávať a organizovať viac a viac textov. Google, Wikipédia, protokoly ako html ktoré nám umožňujú prispievať obsah na internet, blogy, magazíny, knihy, noviny, väčšina obsahu na sociálnych sietiach a na internete atď. sú tvorené veľkým množstvom textov.

Nie len digitálne ale aj mnoho klasických ľudských aktivít v sebe zahŕňa tvorbu alebo vstrebávanie nejakých druhov textov či je to obyčajná medziľudská komunikácia alebo písanie na papier.

Práca s textom sa týka mnohých oborov od študentov, sekretárok až po softvérových architektov a manažérov. Aby veľké množstvo informácií nespôsobilo informačné preťaženie je treba prácu s textom automatizovať a vysporiadať sa s filtrovaním užitočných informácií od tých zbytočných, čo vedie k uľahčeniu práce mnohých ľudí.

Väčšina týchto textov je v podobe prirodzeného jazyka(slovenčina, angličtina) a majú osobnú povahu, čo môže predstavovať prekážku pri automatizácii. Lúdia veľmi jednoducho vedia rozlíšiť zlý text od dobrého.

Manuálne písanie textov môže byť náročné na realizáciu, jednak kvôli potrebe vedomostí v danej oblasti druhak, že to môže byť nudné a cenovo nevýhodné a v nektorých prípadoch, kedy je textu veľmi veľa, aj nemožné. Aj ľudia aj stroje by profitovali z automatizácie týchto činností. Zautomatizovanie takýchto činnosti sa zaoberá NLG alebo generovanie prirodzeného jazyka.

V tejto práci sa zameriame na generovanie textov konkrétne v jazyku slovenskom a anglickom. Kvôli tomu, že väčšina automatizovaní tvorenia textov je robená v angličtine. Predstavuje to niekoľko prekážok. Slovenkých textov je pomerne málo a ešte ťažšie sa získavajú, ak chceme použiť nejaký novší prístup potrebujeme veľa
vzorov na to ako text správne písať(nechceme to robiť manuálne). Navyše slovenčina je morfologicky bohatší jazyk, používa veľa prípon, predpôn a má viac časov ako angličtina. Taktiež exituje málo systémov, ktoré sa o generovanie slovenského textu pokúsili, preto budeme používať niečo, čo funguje na angličtinu ale vieme to použiť aj na slovenčinu. Z tohto vypláva aj ďalší cieľ práce a to porovnanie nejakých aktuálnych prístupov na generovanie textu v slovenčine a angličtine.

Porovnanie chceme robiť kvôli tomu, že systémy majú svoj limit a tým pádom že slovenčina je pomerne expresívny a syntakticky bohatý jazyk ako angličtina. Preto prístupy vhodne pre angličtinu alebo iný jazyk nemusia nutne byt dobré pre slovenčinu, respektíve chceme sa dozvedieť ci je nejaký signifikantný rozdiel medzi rovnakým systémom, ktorý sa učí iný jazyk.

Prínosom práce ma byť porovnanie toho či prístupy použite v iných jazykoch fungujú aj na slovenčine a či sa dá niečo zmysluplné vygenerovať. Na ohodnotenie budeme potrebovať ľudí, ktorý text ohodnotia a povedia či je dobrý na základe rôznych metrík. Dôvod pre použitie ludi je ten, že text je komplexný a automatické overenie správnosti je subjektívne, komplikované a nástroje, ktoré sú prístupné sú nepresné môžu považovať aj zlé texty za správne. 

Môžeme nájsť viacero definícií NLG jednou z klasických je, že NLG je
jednou z oborov umelej inteligencie a NLP, ktorej úlohou je zvyčajne z
nejazyčných vstupov vygenerovať text pre rôzne domény a oblasti
ľudských činnosti \cite{reiter_dale_2000_buildingnlgsystems}.

V poslednej dobre sa však na rôzne úlohy hlavne text-to-text používajú
prístupy, ktorých vstupom nie sú len dáta ale aj
text\cite{Nallapati_2016_sum_s2s, Nisioi_2017_simpl_expl,
Mallinson_2017_para_w_mt}. Rozdiel medzi touto prvotnou definíciou NLG
a reálnymi aplikáciami v súčasnosti značí, že NLG prešlo počas
posledných dvoch desaťročí mnohými zmenami.

Vo všeobecnosti ide o generovanie textu, ktorý je priamo určený na
čítanie pre ľudí. Táto definícia je všeobecná a preto má NLG široké
využitie. Niektoré príklady využitia NLG zahŕňajú napr. generovanie
článkov do novín ako napríklad futbalové
reportáže\cite{vanderlee_krahmer_wubben_2017_PASS}, zdravotnícke
správy, ale aj všeobecnejšie, nezávislé od domény využitia ako
simplifikácia alebo sumarizácia textu, generovanie parafráz,
prekladanie textov\cite{gatt_2018_survey}.
