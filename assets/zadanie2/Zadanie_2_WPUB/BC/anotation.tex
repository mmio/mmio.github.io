\newpage
\thispagestyle{plain}
\begin{center}
\begin{Large}
\textbf{Anotácia} \\
\end{Large}
\end{center}
Slovenská technická univerzita v Bratislave \\
FAKULTA INFORMATIKY A INFORMAČNÝCH TECHNOLÓGIÍ \\
\noindent
Študijný program: \Program \\
\noindent
Autor: \Author \\
\ifthenelse {\boolean{bachelor}}
{
	{Bakalárska práca: }\Title \\
}
{
	{Diplomová práca: }\Title \\
}
Vedúci práce: \Supervisor \\
\Month{ }\Year \\
\noindent
\\

V tejto práci sa budeme zaoberať generovaním textov pre ľudí, ktoré sú určený na čítanie a majú podobu prirodzeného jazyka. Práca obsahuje analýzu oblasti generovania textov prirodzeného jazyka a jednotlivých využití generovaných textov ako sumarizácia, simplifikácia, parafrázovanie a generovanie dialógov, ich vývoj a súčasný stav pre tieto úlohy.

Súčasťou práce je aj návrh a implementácia modelu na generovanie textu na ktoré budú použité hlboké rekurentné neurónové siete, ktoré dosahujú v poslednej dobe úspech v tejto oblasti a zďaleka najlepšie výsledky. Samotné generovanie textu bude robené v jazyku slovenskom a anglickom, účelom implementácie modelu v dvoch jazykoch je hlavne porovnanie rôznych prístupov pre rôzne jazyky.

Všetky experimenty budú robené na paralelných korpusoch podobnej veľkosti a z podobných voľne prístupných zdrojov ako napr. Wikipédia. Evaluácia výsledkov experimentov bude robená manuálne ľudskou silou metódou hodnotenia metrík generovaného text. Dôvodom pre manuálne hodnotenie je, že ľuďmi čitateľný text je komplexný a automatizované hodnotenie nie je vždy vhodné.

\newpage
\blankpage

\thispagestyle{plain}
\begin{center}
\begin{Large}
\textbf{Annotation} \\
\end{Large}
\end{center}
Slovak University of Technology Bratislava \\
FACULTY OF INFORMATICS AND INFORMATION TECHNOLOGIES \\
\noindent
Degree Course: Information systems \\
\noindent
Author: \Author \\
\ifthenelse {\boolean{bachelor}}
{
	{Bachelor thesis: }\mbox{Generating Slovak Texts}\\
}
{
	{Master's thesis: }\mbox{} \\
}
Supervisor: \Supervisor \\
\Year, May \\
\noindent
\\

The goal of this works is to generate text in a natural language which is readable to humans, it contains an analysis of the natural language generation field and its uses like text summarization, simplification, paraphrasing and dialog generation and their evolution.

Part of this work is a design and implementation of a model for text generation. Deep recurent neural networks will be used for that task, the reason for that is their efficiency and the fact the most state-of-the art models use them. Texts will be generated in Slovak and English languages to demostrate differences between those languages from the context of NLG.

All models used will be trained on a paralel corpuses of similar size for example from Wikipedia. The evaulation of the results will be done manualy with human power by evaluating metrics. The reason for that is mainly the complexity of natural languags where automated evaluation is not suficient.

\newpage
\blankpage